\documentclass{beamer}
%
% Choose how your presentation looks.
%
% For more themes, color themes and font themes, see:
% http://deic.uab.es/~iblanes/beamer_gallery/index_by_theme.html
%
\mode<presentation>
{
  \usetheme{default}      % or try Darmstadt, Madrid, Warsaw, ...
  \usecolortheme{default} % or try albatross, beaver, crane, ...
  \usefonttheme{default}  % or try serif, structurebold, ...
  \setbeamertemplate{navigation symbols}{}
  \setbeamertemplate{caption}[numbered]
  \setbeamertemplate{footline}[frame number]
} 

\usepackage[english]{babel}
\usepackage[utf8x]{inputenc}

\title[2016-03-14-scaroot]{Accessing ROOT and C++ functions in Spark}
\author{Jim Pivarski}
\institute{Princeton University --- DIANA}
\date{March 14, 2016}

\xdefinecolor{darkblue}{rgb}{0.1,0.1,0.7}

\begin{document}

\begin{frame}
  \titlepage
\end{frame}

% Uncomment these lines for an automatically generated outline.
%\begin{frame}{Outline}
%  \tableofcontents
%\end{frame}

\begin{frame}{}
My work on integrating ROOT and Spark has split into:
\begin{description}
\item[ScaROOT:] Call ROOT functions (and arbitrary C++) from Scala, and therefore Spark, Hadoop, etc.
\item[root2avro:] Bulk data flow from ROOT files to other file formats or streaming into Spark, Hadoop, etc.
\end{description}

\vfill
For the last few weeks, \textcolor{darkblue}{root2avro} has been my main focus, but I recently got \textcolor{darkblue}{ScaROOT} into usable shape.

\vfill
\textcolor{darkblue}{root2avro} isn't a special case of \textcolor{darkblue}{ScaROOT} for reasons you'll see in a moment.
\end{frame}

\begin{frame}{}
Documented with unit tests and examples on the GitHub wiki page (\url{https://github.com/diana-hep/scaroot}).

\vspace{0.5 cm}
\includegraphics[width=\linewidth]{wiki.png}
\end{frame}

\begin{frame}{Interface}

\begin{block}{}
\vspace{-\baselineskip}
PyROOT dynamically makes proxies to ROOT functions when they're called, so that working in Python is approximately the same as working in CINT/Cling (replacing ``{\tt ->}'' and ``{\tt ::}'' with ``{\tt .}'').
\end{block}

\begin{block}{}
\vspace{-\baselineskip}
Scala (and Java) are compiled languages, so field names of classes have to be declared in advance. They can define classes on the fly (in a private ClassLoader), but these classes can only be used in the main program if they adhere to a predefined interface.
\end{block}
\end{frame}

\end{document}
